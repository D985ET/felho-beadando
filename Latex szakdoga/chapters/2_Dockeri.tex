Alapvetően 2 dolgot dockerizáltam.
\begin{enumerate}
	\item \textbf{Sprinboot} applikációt, ami az interface-t fogja szolgáltatni számunkra.
	\item \textbf{Prometheust} amely segítségével pedig metrikákat tudunk készíteni a későbbi riasztások készítése érdekében.
\end{enumerate}
Ezen alkalmazások dockerizálásához 3 komponensre van szükségünk, egy \textit{prometheus.yml} fájlra, melyben a prometheushoz tartozó configurációs beállításokat helyezzük el.
Továbbá, szükségünk van egy \textit{docker-compose.yml} fájlra is, melyben megadjuk a springboot service-nek is meg a prometheusnak is a portokat és azt, hogy hol érjük el a prometheus.yml, configurációs fájlt is.
Végül, magát a \textit{Dockerfile-t} amelyben az initializációs konfigurációs elemeket adjuk meg.
\subsection{docker-compose.yml}
Ebben a fájlban kettő szolgáltatást kell meghírdetnünk a springbootot és a prometheust.
A sprinboot-ot a 8080-as porton hírdetjük meg és azonos hálózatba kötjük a prometheussal.
A prometheust pedig a 9090-es porton hírdetjük meg, és ugyanúgy egy hálózatba kötjük őket.
Végül létrehozzuk ezt a közös hálózatot.
docker-compose.yml tartalma:
\begin{lstlisting}
	services:
		springboot_service:
			build: .
			restart: always
			ports:
				- 8080:8080
			networks:
				- localprom
		prometheus:
			image: prom/prometheus:v2.44.0
			container_name: prometheus
			ports:
				- "9090:9090"
			networks:
				- localprom
			volumes:
				- ./monitoring/prometheus/prometheus.yml:/etc/prometheus/prometheus.yml
	networks:
		localprom:
\end{lstlisting}
\subsection{prometheus.yml}
Ebben a fájlban, megadjuk a prometheus applikáció nevét, webes elérési útját (/actuator/prometheus), az adatgyűjtések közötti eltelt intervallumot ami 3 másodperc. Majd meghírdetjük a Java applikációnak a prometheust.

prometheus.yml tartalma:
\begin{lstlisting}
	scrape_configs:
		- job_name: 'MyAppMetrics'
		metrics_path: '/actuator/prometheus'
		scrape_interval: 3s
		static_configs:
			- targets: ['host.docker.internal:8080']
			labels:
				application: 'My Spring Boot Application'
\end{lstlisting}

\subsection{Dockerfile}
Alapvetően a Dockerfileban adjuk meg a kezdeti konfigurációkat, mivel induljon el az alkalmazás, milyen packageket futtasson le.
A \textbf{FROM} részben megadjuk a Java verzióját és a projekt kezelőjének a típusát amely a maven.
Ezt követően a \textbf{src} mappát bemásoljuk a docker által készített /home/app/src mappába, melyet követ a Java fájl konfigurációs fájljának az átmásolása a \textit{pom.xml}.
Majd lefuttatjuk a clean package commandot amely kitisztítja a pom.xml fájlt. Majd kihírdetjük az applikációt a 8080-as porton. Melyet követ a jar fájl bementi pontjának a meghatározása.
\\
Dockerfile tartalma:
\begin{lstlisting}
	FROM maven:3.8.3-openjdk-17 AS build
	COPY src /home/app/src
	COPY pom.xml /home/app
	RUN mvn -f /home/app/pom.xml clean package
	EXPOSE 8080
	ENTRYPOINT ["java","-jar","/home/app/target/spring_prom_app.jar"]
\end{lstlisting}
